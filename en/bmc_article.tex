%% BioMed_Central_Tex_Template_v1.06
%%                                      %
%  bmc_article.tex            ver: 1.06 %
%                                       %

%%IMPORTANT: do not delete the first line of this template
%%It must be present to enable the BMC Submission system to
%%recognise this template!!

%%%%%%%%%%%%%%%%%%%%%%%%%%%%%%%%%%%%%%%%%
%%                                     %%
%%  LaTeX template for BioMed Central  %%
%%     journal article submissions     %%
%%                                     %%
%%          <8 June 2012>              %%
%%                                     %%
%%                                     %%
%%%%%%%%%%%%%%%%%%%%%%%%%%%%%%%%%%%%%%%%%


%%%%%%%%%%%%%%%%%%%%%%%%%%%%%%%%%%%%%%%%%%%%%%%%%%%%%%%%%%%%%%%%%%%%%
%%                                                                 %%
%% For instructions on how to fill out this Tex template           %%
%% document please refer to Readme.html and the instructions for   %%
%% authors page on the biomed central website                      %%
%% http://www.biomedcentral.com/info/authors/                      %%
%%                                                                 %%
%% Please do not use \input{...} to include other tex files.       %%
%% Submit your LaTeX manuscript as one .tex document.              %%
%%                                                                 %%
%% All additional figures and files should be attached             %%
%% separately and not embedded in the \TeX\ document itself.       %%
%%                                                                 %%
%% BioMed Central currently use the MikTex distribution of         %%
%% TeX for Windows) of TeX and LaTeX.  This is available from      %%
%% http://www.miktex.org                                           %%
%%                                                                 %%
%%%%%%%%%%%%%%%%%%%%%%%%%%%%%%%%%%%%%%%%%%%%%%%%%%%%%%%%%%%%%%%%%%%%%

%%% additional documentclass options:
%  [doublespacing]
%  [linenumbers]   - put the line numbers on margins

%%% loading packages, author definitions

%\documentclass[twocolumn]{bmcart}% uncomment this for twocolumn layout and comment line below
%\documentclass{bmcart}
\documentclass[doublespacing]{bmcart}

%%% Load packages
%\usepackage{amsthm,amsmath}
%\RequirePackage{natbib}
%\RequirePackage[authoryear]{natbib}% uncomment this for author-year bibliography
%\RequirePackage{hyperref}
\usepackage[utf8]{inputenc} %unicode support
%\usepackage[applemac]{inputenc} %applemac support if unicode package fails
%\usepackage[latin1]{inputenc} %UNIX support if unicode package fails

\usepackage{url}

\usepackage{graphicx}
\graphicspath{{./figuras/}}

%%%%%%%%%%%%%%%%%%%%%%%%%%%%%%%%%%%%%%%%%%%%%%%%%
%%                                             %%
%%  If you wish to display your graphics for   %%
%%  your own use using includegraphic or       %%
%%  includegraphics, then comment out the      %%
%%  following two lines of code.               %%
%%  NB: These line *must* be included when     %%
%%  submitting to BMC.                         %%
%%  All figure files must be submitted as      %%
%%  separate graphics through the BMC          %%
%%  submission process, not included in the    %%
%%  submitted article.                         %%
%%                                             %%
%%%%%%%%%%%%%%%%%%%%%%%%%%%%%%%%%%%%%%%%%%%%%%%%%


%\def\includegraphic{}
%\def\includegraphics{}



%%% Put your definitions there:
\startlocaldefs
\endlocaldefs


%%% Begin ...
\begin{document}

%%% Start of article front matter
\begin{frontmatter}

\begin{fmbox}
\dochead{Research}

%%%%%%%%%%%%%%%%%%%%%%%%%%%%%%%%%%%%%%%%%%%%%%
%%                                          %%
%% Enter the title of your article here     %%
%%                                          %%
%%%%%%%%%%%%%%%%%%%%%%%%%%%%%%%%%%%%%%%%%%%%%%

\title{Performance of Eventual Consistency and Per-record Timeline Consistency 
in Geo-replicated Systems}

%%%%%%%%%%%%%%%%%%%%%%%%%%%%%%%%%%%%%%%%%%%%%%
%%                                          %%
%% Enter the authors here                   %%
%%                                          %%
%% Specify information, if available,       %%
%% in the form:                             %%
%%   <key>={<id1>,<id2>}                    %%
%%   <key>=                                 %%
%% Comment or delete the keys which are     %%
%% not used. Repeat \author command as much %%
%% as required.                             %%
%%                                          %%
%%%%%%%%%%%%%%%%%%%%%%%%%%%%%%%%%%%%%%%%%%%%%%

\author[
   addressref={aff1},                   % id's of addresses, e.g. {aff1,aff2}
   corref={aff1},                       % id of corresponding address, if any
%   noteref={n1},                        % id's of article notes, if any
   email={mdediana@ime.usp.br}   % email address
]{\inits{MDD}\fnm{Mauricio} \snm{De Diana}}
\author[
   addressref={aff1},
   email={gerosa@ime.usp.br}
]{\inits{MAG}\fnm{Marco Aurélio} \snm{Gerosa}}

%%%%%%%%%%%%%%%%%%%%%%%%%%%%%%%%%%%%%%%%%%%%%%
%%                                          %%
%% Enter the authors' addresses here        %%
%%                                          %%
%% Repeat \address commands as much as      %%
%% required.                                %%
%%                                          %%
%%%%%%%%%%%%%%%%%%%%%%%%%%%%%%%%%%%%%%%%%%%%%%

\address[id=aff1]{%                           % unique id
  \orgname{Computer Science Department, University of São Paulo (USP)}, % university, etc
  \street{R. do Matão, 1010 - Vila Universitária, São Paulo - SP},                     %
  %\postcode{05508-00}                                % post or zip code
  \city{São Paulo},                              % city
  \cny{Brazil}                                    % country
}
%\address[id=aff2]{%
%  \orgname{Marine Ecology Department, Institute of Marine Sciences Kiel},
%  \street{D\"{u}sternbrooker Weg 20},
%  \postcode{24105}
%  \city{Kiel},
%  \cny{Germany}
%}

%%%%%%%%%%%%%%%%%%%%%%%%%%%%%%%%%%%%%%%%%%%%%%
%%                                          %%
%% Enter short notes here                   %%
%%                                          %%
%% Short notes will be after addresses      %%
%% on first page.                           %%
%%                                          %%
%%%%%%%%%%%%%%%%%%%%%%%%%%%%%%%%%%%%%%%%%%%%%%

\begin{artnotes}
%\note{Sample of title note}     % note to the article
%\note[id=n1]{Equal contributor} % note, connected to author
\end{artnotes}

\end{fmbox}% comment this for two column layout

%%%%%%%%%%%%%%%%%%%%%%%%%%%%%%%%%%%%%%%%%%%%%%
%%                                          %%
%% The Abstract begins here                 %%
%%                                          %%
%% Please refer to the Instructions for     %%
%% authors on http://www.biomedcentral.com  %%
%% and include the section headings         %%
%% accordingly for your article type.       %%
%%                                          %%
%%%%%%%%%%%%%%%%%%%%%%%%%%%%%%%%%%%%%%%%%%%%%%

\begin{abstractbox}

\begin{abstract} % abstract
Large-scale web systems use data replication among data centers to achieve high
levels of performance and availability, trading off these attributes against
consistency.  Consistency models define the consistency guarantees that a system
provides and its trade-offs.  Eventual consistency is a more relaxed consistency
model, widely deployed in large-scale systems. By its turn, the less popular
per-record timeline consistency strikes a better balance when requirements over
consistency are more restrictive. However, to be able to make a decision,
developers need to understand its performance penalties and the situations when
they apply. 

Here we show that under certain workloads and network conditions, a storage
system exhibits similar performance for both consistency models. The
experimental results confirm that a system using per-object timeline consistency
and exhibiting high data locality and read/write ratios suffer a low performance
hit compared to the same system using eventual consistency. This comparative
helps developers and system administrators on cost estimates and capacity
planning of large-scale systems.
%\parttitle{First part title} %if any
%Text for this section.

%\parttitle{Second part title} %if any
%Text for this section.
\end{abstract}

%%%%%%%%%%%%%%%%%%%%%%%%%%%%%%%%%%%%%%%%%%%%%%
%%                                          %%
%% The keywords begin here                  %%
%%                                          %%
%% Put each keyword in separate \kwd{}.     %%
%%                                          %%
%%%%%%%%%%%%%%%%%%%%%%%%%%%%%%%%%%%%%%%%%%%%%%

\begin{keyword}
\kwd{distributed systems}
\kwd{consistency models}
\kwd{large scale web systems}
\kwd{performance analysis}
\end{keyword}

% MSC classifications codes, if any
%\begin{keyword}[class=AMS]
%\kwd[Primary ]{}
%\kwd{}
%\kwd[; secondary ]{}
%\end{keyword}

\end{abstractbox}
%
%\end{fmbox}% uncomment this for twcolumn layout

\end{frontmatter}

%%%%%%%%%%%%%%%%%%%%%%%%%%%%%%%%%%%%%%%%%%%%%%
%%                                          %%
%% The Main Body begins here                %%
%%                                          %%
%% Please refer to the instructions for     %%
%% authors on:                              %%
%% http://www.biomedcentral.com/info/authors%%
%% and include the section headings         %%
%% accordingly for your article type.       %%
%%                                          %%
%% See the Results and Discussion section   %%
%% for details on how to create sub-sections%%
%%                                          %%
%% use \cite{...} to cite references        %%
%%  \cite{koon} and                         %%
%%  \cite{oreg,khar,zvai,xjon,schn,pond}    %%
%%  \nocite{smith,marg,hunn,advi,koha,mouse}%%
%%                                          %%
%%%%%%%%%%%%%%%%%%%%%%%%%%%%%%%%%%%%%%%%%%%%%%

%%%%%%%%%%%%%%%%%%%%%%%%% start of article main body
% <put your article body there>

\section{Introduction}

To continuously serve users spread around the globe with low response times,
large-scale web systems replicate data across several servers, which servers are
spread over multiple data centers located in different geographical locations,
linked by wide area networks (WANs). Such structure poses challenges to
maintaining the consistency among the replicas, mainly due to network latencies
and node / network failures.

Developers and administrators of such systems need to seek a balance between
consistency, performance and availability. Several consistency models have been
described in the literature, each of them defining a different set of
conditions, guarantees and trade-offs. A common decision in large-scale systems
is the use of more relaxed consistency models to achieve low latency and high
availability.  However, inconsistencies are more likely to happen with more
relaxed consistency models.

Eventual consistency is a more relaxed consistency model which has become
popular after the publication of the Dynamo paper \cite{DeCandia2007}. Eventual
consistency guarantees that, once updates cease, all replicas will converge to
the same value at some moment in the future. As long as updates are occurring,
replicas may be inconsistent and clients may read stale or divergent values
depending on which replica processes a given request. From that comes the need
for the application to implement conflict resolution mechanisms and compensation
actions to handle the results of actions taken based on inconsistent state. The
added complexity negatively affects code maintainability and system operations.

Some web applications strictly require high availability levels. For example,
Amazon needs its users to be able to add items to their shopping cart even if it
leads to some error that must be handled afterwards \cite{DeCandia2007}.

However, eventual consistency is not appropriate for every case, some systems
need stronger consistency models. An example is a bidding application that
cannot allow conflicts on bidding history. If eventual consistency is used and a
network partition happens, users bidding in the different sides of the partition
will lead up to divergent bidding histories. Furthermore, developers of
applications with less strict availability requirements can benefit from simpler
assumptions when programming.

Per-object timeline consistency is a middle ground between linearizability
\cite{Herlihy1990} and eventual consistency. It was first implemented in the
PNUTS database system, developed by Yahoo! \cite{Cooper2008}. Under this
consistency model, each object has a master replica, which is the only replica
allowed to apply updates. By serializing updates in the master replica,
conflicts among replicas cannot happen since every replica receive the updates
in the same order. Moreover, the most recent value is known at any time, it is
the value stored in the master replica. However, since updates are asynchronous,
stale reads due to latency can happen. Objects are versioned and read requests
carry the required version, which can be a specific version, the most recent
(served exclusively by the master) or any version. The main drawback of
per-record timeline consistency is that updates and most recent reads become
unavailable in case of a failure preventing the communication between a client
and the master replica.

Per-object timeline consistency may be a better fit than eventual consistency
for web applications where users tolerate some level of unavailability. Also,
besides availability, performance must also be considered. The main performance
hit in per-record timeline consistency is due to the cost of updates and
consistent reads when the request must traverse a WAN to reach the master
replica in another data center. However, specific conditions may alleviate this
issue, such as high access locality rates and high read/write ratios. In fact,
PNUTS authors observed up to 85\% locality rate and 94\% of read/write ratio in
a study of Yahoo's applications \cite{Cooper2008, Kadambi2011}. Based on that,
they implemented heuristics that improve per-record timeline consistency
performance.

This study compares the performance of a geo-replicated storage system using
eventual consistency and per-record timeline consistency under different
workloads and network conditions. The results make a useful resource for
developers and system administrators estimating system development costs and
doing capacity planning.

%% ------------------------------------------------------------------------- %%
\section{Eventual Consistency and Per-record Timeline Consistency}

Eventual consistency guarantees that once updates cease, all replicas will converge to the same value. While updates are happening, replicas might be
inconsistent and clients may access outdated or divergent data -- consequently,
the system must implement algorithms for conflict detection and resolution. A system may use quorums
to reduce the chances of conflicts, the drawback is decreased availability when
a specific quorum can not be reached due to node or network failures \cite{Vogels2009}.
Quorums are commonly defined by the \textit{N}, \textit{R} and \textit{W} parameters:
\textit{N} is the replication factor and represents the number of replicas of a given object in the system.
\textit{R} and \textit{W} are respectively the amount of replicas that need
to agree on the same value for a read and for a write to be successful.
$N < R + W$ eliminates the possibility of
clients reading inconsistent data, since there is an intersection between the
subsets of replicas for reading and writing.

Reasoning about a web application is simpler under a more strict consistency model. For
example, an auction site cannot tolerate conflicts in the bidding history of a
product. In a system using eventual consistency, in case of a failure that
splits the network in two or more partitions, users acting in each partition will have their
own view of the bidding history, what is equivalent to two or more simultaneous auctions
for the same item.

Per-object timeline consistency trades availability for consistency in some situations
\cite{Cooper2008}. For each stored object, it allows updates only to one of its
replicas, the master replica. Due to asynchronous replication, replicas may have
different values due to network latency or network failure, however, at any given moment,
the most up-to-date replica is known. In every read, clients choose whether to
accept only the most recent, any value or a specific version of the object. Due to
the master replica establishing an update ordering to the other replicas, divergences do
not happen and conflict detection and resolution mechanisms are not needed. The
main drawback of per-record timeline consistency is that the use of a master replica
leads to unavailability of writes and most recent reads in the event of a node or network failure
that renders it unreachable.

The highest impact factor in the performance of per-record timeline consistency is the fact
that consistent operations that are not made in the master replica's data center
incur the cost of communication over a WAN.  However, the authors of PNUTS
point that some applications on Yahoo! exhibit access locality up to 85
\% and write / read ratio as low as 6\% \cite{Kadambi2011, Cooper2008}. 
A social network application fits this pattern, most users access the application from the
same geographic location and a post or picture is written once (or edited few times at most)
but read several times. Thus, in an application in which
the amount of reads is much greater than the amount of writes, network
communication costs are low, especially if reads do not necessarily need the
latest value.
The authors of PNUTS implemented a simple heuristic by which the master replica migrates to the
data center that processed the three latest writes of the object. Back to the social network example, this 
would be the case of a user moving to another country, for example, and starting accessing the system from
the new place.

%% ------------------------------------------------------------------------- %%
\section{Experiment Planning}

We performed an experiment to compare the performance of eventual consistency and per-object timeline consistency under different situations. Planning and execution used \cite{Jain1991} as the main
methodological reference.

The three techniques available for system performance analysis are simulation,
analytical modeling and measurement \cite{Jain1991}. We picked measurement, as 33 parameters were
initially considered for this study (see following section), simulators and models were
discarded because the needed simplifications to apply them would result in loss of accuracy.

To be able to accurately compare the consistency models without interference
of software architecture and technology stack factors, the experiment should
make use of a single system with both consistency models. As no such system has
been found, we implemented per-record timeline consistency
in Riak, a free software system which already implements eventual consistency
\cite{Riak2013}. Besides the new consistency model, we also implemented a
partitioning algorithm that ensures that there exist at least one replica of
each object in each data center \cite{DeDiana2013a, DeDiana2013b}.

For running the experiments and gathering results, we used the Riak benchmark 
Basho Bench \cite{BashoBench2013}. We adapted it to be able to run more than
a single instance simultaneously, with one instance for each data center.

To achieve experiment reproducibility, the use of a real WAN was avoided.
Instead, the experiments emulated a WAN with the Linux tools
tc (traffic control) / netem (Network Emulation). These tools provide functionality
for emulating network characteristics such as latency and packet
loss. Network settings in Linux were configured according to
recommended optimizations for WAN communication \cite{ESnet2012}, for example,
the use of the double of the Bandwidth-Delay Product (BDP) as transmission and
reception buffer size.

The experiments were carried out on Grid'5000, a platform for creation, implementation
and monitoring of parallel and distributed system experiments
\cite{Grid50002013}.

The study used factorial design, which consist of a combination of factors in
each experiment comprising the study \cite{Jain1991}. The more
factors and levels in a study, the more resources are needed to its execution.
Moreover, usually a few factors explain the most effects on the response. Therefore,
a selection of the most influential factors was performed by the use of
2\textsuperscript{k} experiments, which consist of the selection of only
the minimum and maximum levels for each factor, resulting in a total of
2\textsuperscript{k} experiments, where \textit{k} is the number of factors.

%% ------------------------------------------------------------------------- %%
\section{Fixed Parameters}

After defining the experimental design, 33 parameters were listed, 16
of which were fixed due to resource constraints. It is worth noting that
LAN parameters were the actual values seeing in the
cluster network, while WAN parameters were emulated. The fixed parameters and their
respective values were:

\begin{itemize}

\item \textbf{Cluster:} The experiments used the cluster \textit{sol} in
Grid'5000. The nodes in the cluster have AMD Opteron 2218 2.6 GHz CPU, 4 GB RAM
and 1 Gb/s network cards.

\item \textbf{Storage engine:} By using memory as storage mechanism we avoided
having to consider the effects of disk and disk cache and the interaction
between disk cache and the amount of RAM, so the only I/O effects observed were
due to the network.

\item \textbf{Data center capacity:} Data centers had the same capacity - the
same number of nodes per data center and nodes with the same hardware
configuration.

\item \textbf{Key partitioning algorithm:} Riak's default algorithm was used
(consistent hashing).

\item \textbf{Replication factor ($ N $):} 3 is the value that results in a
reasonable balance between performance, availability and durability in real
applications \cite{DeCandia2007}.

\item \textbf{Migration threshold (for per-record timeline consistency):} 3 is PNUTS
default value \cite{Cooper2008}.

\item \textbf{Interface access:} HTTP, due to its simplicity.

\item \textbf{Log level:} WARN, since exploratory experiments showed loss of
performance when the logging level was INFO.

\item \textbf{Hardware configuration of intermediary network devices:} The
single network device was a FastIron Super X switch, tests have shown there were
no bottlenecks in the switch even in experiments with high consumption of bandwidth.

\item \textbf{Network topology:} Star, the only available network topology in
the cluster.

\item \textbf{LAN bandwidth:} 1 Gb/s was the bandwidth available in the node
network cards.

\item \textbf{LAN latency:} $167\mu$s latency cluster measured with 60 samples
of pings spaced by five seconds.

\item \textbf{LAN jitter:} $90\mu$s, jitter measured by the same means as
latency.

\item \textbf{WAN bandwidth:} 100 Mb/s, based on informal study citing this is
the commonly observed  bandwidth between AWS availability zones
\cite{Pujol2012}.

\item \textbf{Number of WAN links:} 1, which means two simulated data centers.

\item \textbf{Request arrival rate:} 15 operations/s for each thread of each
benchmark instance.

\end{itemize}

There were 17 remaining candidate factors. Three of those correspond to the mode
factor.

%% ------------------------------------------------------------------------- %%
\section{Mode}

Three factors have received different treatment during the experiments: consistency
model, replication configuration (for eventual consistency) and required version for reads (for per-record timeline consistency). This was done because the combinations of
these factors define storage system configurations that result in different
local/remote request ratios. Thus, these factors were treated as a single factor,
\textit{mode}. The modes were:

\begin{itemize}

\item \textit{ev1}: Eventual consistency with $W$ = 1 and $R$ = 1

\item \textit{ev2}: Eventual consistency with $W$ = 2 and $R$ = 1

\item \textit{any}: Per-object timeline consistency with reads of any version

\item \textit{lat}: Per-object timeline consistency with latest version reads

\end{itemize}

Whereas the replication factor was set to 3 and the presence of at least one replica
on each data center was guaranteed, from the client standpoint, two situations were possible with respect to the location
of replicas: one local and two remote or vice versa. Given that,
\textit{ev1} resulted in local reads and writes. Mode \textit{ev2} resulted in
local reads and half remote, half local writes. Mode \textit{any} resulted in
local reads and the amount of local/remote writes depending on access locality.
Finally, \textit{lat} resulted in both reads and writes being dependent on location.

These modes involve trade-offs beyond performance and consistency. The main is
durability which to \textit{ev2} is higher than for other cases in which write
confirmation of a single replica is sufficient.

After defining the mode factor, 14 candidate factors remained, too many for the final study. We performed 2\textsuperscript{k} experiments to reduce this amount in an informed way.

%% ------------------------------------------------------------------------- %%
\section{Selection of Factors}

An approach for the selection of factors would group all candidates factors in a
single 2\textsuperscript{k} experimental study. The problem is that even with
only two levels by factor, 2\textsuperscript{14} experiments was still a prohibitive number.

We splitted the experiments into smaller groups of related factors and
ran experiments for each group separately. With that, comparison between factors of
different groups and their interactions was lost, however, since most factors
have shown little influence inside their groups, as shown in the following
subsections, this approach did not present a threat to validity.

Most factors were prone to interactions with network factors. The WAN latency in
particular had been very influential in exploratory studies, fact later confirmed by
the study of network factors. Given that, the adopted approach was the use of
WAN latency as a representative of network factors when necessary.

Cases happened where the results of all experiments of a study were
similar, regardless of the factor levels. To handle these cases the coefficients
of variation (CVs) of the responses were calculated to estimate the influence of
that set of factors and interactions as a whole. Thus, a low CV indicated that
none of these factors were influential.

For per-record timeline consistency, only the insertion of objects in the load stage was
not enough for the system to operate at its steady state during the experiments.
This happens because, at the end of the load, each object in the database has
received a single access from each data center, no master replica would have
migrated yet. Therefore, we implemented a warmup phase between data load and the experiment run according to the locality of the experiment.

Four intermediate studies were conducted, they are described in the following
subsections. Whenever mode and locality had to be fixed, they were respectively
at \textit{lat} and 50\%, values that result in a balanced amount of reads and
writes, both local and remote. When necessary, latency was fixed at 100 ms. In
the results, the 10 and 90 percentiles respectively represent local and remote
requests.

%% ------------------------------------------------------------------------- %%
\subsection{System and benchmark size factors}

The number of nodes and benchmark instances not only influenced the
responses, it also influenced operational issues related to node reservation --
by Grid'5000 rules, the greater the number of reserved nodes, the smaller the
reservation time. Therefore, a study was done to determine the influence of
these factors. We used the following levels, the value in parentheses
being the identifier of the factor in the tables with results below:

\begin{itemize}

\item Number of system nodes (N): 8 and 16

\item Number of benchmark instances (B): 2 and 4

\item Number of threads in each benchmark instance (T): 32 and 64

\end{itemize}

The result of the study is shown in Table
\ref{tab:estudo_para_fatores_de_tamanho_do_sistema}. The system size had greater
influence on the results and the number of benchmark instances and threads were
not negligible, specially considering the interactions between them.
Nevertheless, these factors were disregarded due to the excessive number of
factors.

\begin{table}[h!]
\caption{Study for system size factors.}
\begin{tabular}{ccccccccc} \hline

Operation & Percentile & N & B & T & NB & NT & BT & NBT\\\hline

read & 10 & 30 & 18 & 22 & 10 & 8 & 7 & 4 \\

read & 90 & 65 & 13 & 15 & 3 & 4 & 0 & 0 \\

write & 10 & 96 & 2 & 1 & 0 & 0 & 0 & 0 \\

write & 90 & 65 & 15 & 13 & 3 & 3 & 0 & 0 \\\hline

\end{tabular}
\label{tab:estudo_para_fatores_de_tamanho_do_sistema} \end{table}

Given the results, the fixed values were:

\begin{itemize}

\item Number of system nodes: 16

\item Number of benchmark instances: 4

\item Number of threads in each benchmark instance: 32

\end{itemize}

These values were selected because they resulted in the ``lightest''
configuration, avoiding network bottlenecks and system overload. Although
desirable, a greater number of nodes would imply lack of hardware homogeneity
and operational difficulties due to the Grid'5000 rules.

%% ------------------------------------------------------------------------- %%
\subsection{Database factors}

The database size affected memory
usage and bandwidth comsumption in the nodes. The levels used in the study were:

\begin{itemize}

\item Number of stored objects (O): 64,000 and 256,000

\item Object sizes (S): 100 and 10,000 bytes

\end{itemize}

The result of the study is shown in Table
\ref{tab:estudo_para_fatores_de_banco_de_dados}. The number of objects did not
affect system performance. Object size did not affect performance of remote
requests, however, they correspond to 100\% of influence regarding local
requests. Nevertheless, the CV of local requests indicated that their influence
was not as great - 19\% for reads and 16\% for writes.

\begin{table}[h!]
\caption{Study for database factors, L is for latency.}
\begin{tabular}{ccccccccc} \hline

Operation & Percentile & O & S & L & OS & OL & SL & OSL\\\hline

read & 10 & 0 & 100 & 0 & 0 & 0 & 0 & 0 \\

read & 90 & 0 & 0 & 100 & 0 & 0 & 0 & 0 \\

write & 10 & 0 & 100 & 0 & 0 & 0 & 0 & 0 \\

write & 90 & 0 & 0 & 100 & 0 & 0 & 0 & 0 \\\hline

\end{tabular}
\label{tab:estudo_para_fatores_de_banco_de_dados} \end{table}

Given the results, the fixed values were:

\begin{itemize}

\item Number of stored objects: 128,000

\item Object sizes: 500

\end{itemize}

The warmup time depended on the number of stored objects, thus the lower the
number, the faster the execution of experiments.  Conversely, a too small number
would result in an excessive number of conflicts.  For the size of stored
objects, the value was chosen based on a study of caching systems on Facebook,
which reports that 90\% of the objects are smaller than 500 bytes
\cite{Atikoglu2012}.

%% ------------------------------------------------------------------------- %%
\subsection{Network factors}

Given the purpose of this work, the study for network factors was one of the
most important for the factor selection. The levels were as follows:

\begin{itemize}

\item WAN latency (L): 100 and 300 ms

\item WAN jitter (J): 1 and 60\%

\item WAN packet loss rate (P): 0.01 and 0.3\%

\item WAN packet duplication rate (D): 0.05 and 5\%

\item WAN packet reordering rate (R): 0.05 and 5\%

\item TCP variant (V): CUBIC and H-TCP

\end{itemize}

Latency levels were based on a study that reports the latencies between regions
of the Amazon Web Services \cite{Sovran2011}, in which the lowest latency
between data centers in each U.S. coast was 82 ms and the highest was 277 ms
between Ireland and Singapore.

Project PingER served as the basis for other factors \cite{PingER2013}. In
January 2013, it shows an average latency of 238.062 ms with standard deviation
142.996 ms, resulting in a jitter of 60\%. The previous 11 months show similar
figures. The median packet loss rate was 0.178\%. The average packet
duplication rate in January 2013 was 0.006\%. The values used in the experiment
for duplication and reordering were higher than those observed by PingER, yet
they did not influence the response.

Both H-TCP and CUBIC were designed with a focus on high bandwidth, high latency
networks (high BDP), and were chosen because they are cited in references on TCP
tuning for WANs \cite{ESnet2012}.

In the emulated network, latency defines the minimum level and jitter the maximum
value that latency can reach. For example, when latency is configured as 100ms and variation
as 60\%, the emulator generates latency values between 100 ms and 160 ms. The
generated values followed the normal distribution within the specified range of
latency.

The result of the study is presented in Table
\ref{tab:estudo_para_fatores_de_rede}, columns of interactions between factors
with all cells less than 1\% were suppressed for the sake of space. The
responses of local requests showed CVs of 1\%, so their rows were also
suppressed -- which indicates that WAN does not affect local requests.

\begin{table}[h!] 
\caption{Network factors study.} \label{tab:estudo_para_fatores_de_rede}
\begin{tabular}{ccccccccc} \hline

Operation & Percentile & L & J & P & D & R & V & LJ\\\hline

read & 90 & 72 & 21 & 1 & 0 & 0 & 0 & 6\\

write & 90 & 69 & 23 & 1 & 0 & 0 & 0 & 6\\\hline

\end{tabular}

\end{table}

Latency, jitter and the first order interaction between them make up for close to 100\% of the response. Therefore, the selected levels for
these factors were:

\begin{itemize}

\item WAN latency (ms): 0, 100, 200 and 300

\item WAN jitter (\%): 0 and 60

\end{itemize}

Latency and jitter at zero are equivalent to having the entire system operating
on a LAN. The results obtained in these cases were used as aid in the
interpretation of other results, but were not considered in the final study.

The fixed values of the disregarded factors were:

\begin{itemize}

\item WAN packet loss rate: 0\%

\item WAN packet duplication rate: 0\%

\item WAN packet reordering rate: 0\%

\item TCP variant (T): CUBIC

\end{itemize}

%% ------------------------------------------------------------------------- %%
\subsection{Workload factors}

This was one of the most important studies along with the network factors one.
The levels used were as follows:

\begin{itemize}

\item Read / write ratio (R): 2:1 and 10:1

\item Object popularity (P): uniform (the request arrival rate average for each
object is the same) and skewed (arrival rate follows a Pareto
distribution)

\item Locality (X): 50\% (no locality) and 90\% (90\% of accesses to a given
object coming from one data center and 10\% from the other)

\end{itemize}

As the modes behave differently depending on locality, we performed one set of experiments for each mode.

The read/write ratio and locality did not use percentiles, they use response
time average of reads and writes instead. The first factor
relates to the composition between reads and writes, while the second alters the
composition between local and remote requests, so these factors do not make
sense in percentiles separated by request type. For example, 50\% locality means
that the 70th percentile is remote requests, while with 90\% locality the same
percentile represents local requests. If the analysis was performed by
percentile, this location information would be lost and locality would never
have any influence.

The result of the study is shown in Table
\ref{tab:estudo_para_fatores_de_carga_de_trabalho}, in which all columns with
all rows with cells smaller than 5\% were removed for the sake of space. The results for local
requests showed CVs around 2\% for all modes, indicating that none of the
factors influenced local requests.

\begin{table}[h!]
\caption{Study for workload factors, L is latency.}
\begin{tabular}{cccccccccc} \hline

Mode & R & X & P & L & RX & RL & XL & PL & XPL\\\hline

\textit{ev1} & 19 & 12 & 2 & 31 & 0 & 2 & 6 & 6 & 8\\

\textit{ev2} & 50 & 0 & 0 & 39 & 0 & 11 & 0 & 0 & 0\\

\textit{any} & 25 & 30 & 0 & 19 & 9 & 6 & 8 & 0 & 0\\

\textit{lat} & 0 & 53 & 0 & 34 & 0 & 0 & 13 & 0 & 0\\\hline

\end{tabular} 
\label{tab:estudo_para_fatores_de_carga_de_trabalho}
\end{table}

As expected, locality and latency influenced responses in general. The impact of
popularity of objects is virtually nil. Although some modes were apparently
impacted by read/write ratio, this impact was a consequence of the relationship
between local and remote requests. For \textit{ev1}, both reads and writes are
local and the read/write ratio and their interactions with other factors have little impact in this mode. For \textit{lat}, reads and writes are local or remote
depending on locality and read/write ratio does not impact this mode. For
\textit{ev2}, all reads are local and half of writes are remote, therefore when
the read/write ratio changes, the ratio between local and remote requests
changes proportionally -- as expected, this mode is impacted by the read/write ratio.
The same observation holds for \textit{any}, which has all reads local and
writes depending on locality, and has also impacted the read/write ratio. The
read/write ratio would likely influence the response if the storage mechanism
was disk instead of memory, since writes would be affected by the write
time to disk, while reads could be faster due to disk caching.

Therefore, only locality was selected as a factor:

\begin{itemize}

\item Locality (\%): 50 and 90

\end{itemize}

The fixed values of the disregarded factors were:

\begin{itemize}

\item Read/write ratio: 2:1

\item Object popularity: uniform

\end{itemize}

%% ------------------------------------------------------------------------- %%
\section{Final Study}

The final study consisted of a total of 64 experiments with the factors selected
by the 2\textsuperscript{k} studies, shown in
Table~\ref{tab:fatores_e_niveis_do_estudo_final}. Two replications of the study were used to
estimate the variability of the experiments, the measured average CVs
were 1\% for reads and 0.8\% for writes.

\begin{table}[h!]
\caption{Factors and levels in the final study.}
\begin{tabular}{lcc} \hline

Factor & Levels & Number of levels\\\hline

Mode & \textit{ev1}, \textit{ev2}, \textit{any} e \textit{lat} & 4\\

WAN latency (ms) & 0, 100, 200 and 300 & 4\\

WAN jitter (\%) & 0 and 60 & 2\\

Locality & 50\% and 90\% & 2\\\hline

\end{tabular}

\label{tab:fatores_e_niveis_do_estudo_final}

\end{table}

The system showed similar behavior for 100~ms, 200~ms and 300~ms latency levels, so we used only one of them on the presentation of the final results. The boxplot for the 200~ms latency level is shown in Figure \ref{fig:boxplot_dos_tempos_de_resposta}.
When a box cannot be seeing in the figure indicate that all requests delimited
by the boxplot whiskers were local. The same analysis was done with maximum load
instead of 15 operations/s per thread. As
Figure~\ref{fig:boxplot_dos_tempos_de_resposta_para_carga_maxima} shows, all
modes showed an increase in response times, but the behavior did not change.

\begin{figure}[h!]
\caption{Boxplot of response times for 15 operations/s per thread.}
\includegraphics[width=1.0\textwidth]{boxplot200.png}
\label{fig:boxplot_dos_tempos_de_resposta}
\end{figure}

\begin{figure}[h!]
\caption{Boxplot of response times for maximum load per thread.}
\includegraphics[width=1.0\textwidth]{boxplot200_max.png}
\label{fig:boxplot_dos_tempos_de_resposta_para_carga_maxima}
\end{figure}

%% ------------------------------------------------------------------------- %%
\section{Threats to Validity}

Some parameters were fixed and influential parameters were disregarded.  Thus,
studies that use other values for the parameters or consider other factors may
yield different results. This is particularly true for the number of system
nodes, which showed as highly influential.  In addition, levels at different ranges
may yield different results \cite{Jain1991}.

The experiments do not take node failures into account.  Experiments with the
system operating in a failure mode (from the failure of a single node to a whole
datacenter) will likely yield to different results. Such situations were not part of the experiments due to resource limitations.

%% ------------------------------------------------------------------------- %%
\section{Related Work}

A common approach in the distributed systems literature is the proposal of a new
concept and the implementation of a system that uses this concept followed by an
analysis of its performance. Both papers on Dynamo \cite{DeCandia2007} and on
PNUTS \cite{Cooper2008} show performance analyses. A second paper about PNUTS analyzes bandwidth consumption
under different replication policies for communication over WANs
\cite{Kadambi2011}.

Several papers present performance analyses of storage systems using replication
over WANs. In most cases, the goals of the systems are to prove ideas other than the
efficiency of their consistency model. COPS uses causal+ consistency
that is similar to causal consistency with more guarantees and implements
transactions \cite{Lloyd2011}. Scatter proposes an scalable architecture for a strong consistent system \cite{Glendenning2011}. Windows Azure provides a
cloud storage system with strong consistency \cite{Calder2011}. Megastore uses
Paxos to implement strong consistency \cite{Baker2011}. None of these works
present comparisons with other systems or other consistency models. As they do
not use a common benchmark or the same environment, it is difficult to compare them.

The different consistency options in Cassandra and its resulting availability and
performance were analyzed in \cite{Beyer2011}, the conclusion is that settings with
more strict consistency models perform worse. Performance and availability of
master-slave replication and chain replication are compared in
\cite{vanRenesse2004}, each of them using strong consistency and eventual
consistency. None of these studies consider operation over WAN or different
workloads.

A closer work to this is a comparison between Cassandra, HBase,
PNUTS and sharded MySQL under different workloads \cite{Cooper2010}. The results
are a comparison between these systems, but do not say much about their
consistency models, since each have different architectures and configuration
options. Moreover, it only considers the systems operating in a LAN, not a WAN.

%% ------------------------------------------------------------------------- %%
\section{Conclusions}

This study compared the performance of a single storage system operating over a
WAN using two different consistency models. It also showed results
about the influence of different factors and their interactions on the system
performance. The results show that per-object timeline consistency is competitive to eventual consistency
when write locality is high and when stale reads are an option.

The main advantages of per-record timeline consistency over eventual consistency are the
guarantee that conflicting updates do not occur and that the location of the
most recent value is known -- the master. An interesting scenario for its use is
the case where the application tolerates staleness in most reads, but in a few
it needs the most recent value. Per-object timeline consistency main drawbacks are that the
unavailability of the master replica prevents writes and ``most recent'' reads from
happening and its relatively high response time variability. For applications where response time
requirements are on the 99.9 percentile, such as Amazon \cite{DeCandia2007},
per-record timeline consistency is not appropriate.

%%%%%%%%%%%%%%%%%%%%%%%%%%%%%%%%%%%%%%%%%%%%%%
%%                                          %%
%% Backmatter begins here                   %%
%%                                          %%
%%%%%%%%%%%%%%%%%%%%%%%%%%%%%%%%%%%%%%%%%%%%%%

\begin{backmatter}

%\section*{Competing interests}
%  The authors declare that they have no competing interests.

%\section{Author's contributions}
%    Text for this section \ldots

\section*{Acknowledgements}
Experiments presented in this paper were carried out using the Grid'5000
experimental testbed, being developed under the INRIA ALADDIN development action
with support from CNRS, RENATER and several Universities as well as other
funding bodies.

%%%%%%%%%%%%%%%%%%%%%%%%%%%%%%%%%%%%%%%%%%%%%%%%%%%%%%%%%%%%%
%%                  The Bibliography                       %%
%%                                                         %%
%%  Bmc_mathpys.bst  will be used to                       %%
%%  create a .BBL file for submission.                     %%
%%  After submission of the .TEX file,                     %%
%%  you will be prompted to submit your .BBL file.         %%
%%                                                         %%
%%                                                         %%
%%  Note that the displayed Bibliography will not          %%
%%  necessarily be rendered by Latex exactly as specified  %%
%%  in the online Instructions for Authors.                %%
%%                                                         %%
%%%%%%%%%%%%%%%%%%%%%%%%%%%%%%%%%%%%%%%%%%%%%%%%%%%%%%%%%%%%%

% if your bibliography is in bibtex format, use those commands:
\bibliographystyle{bmc-mathphys} % Style BST file (bmc-mathphys, vancouver, spbasic).
\bibliography{bmc_article}      % Bibliography file (usually '*.bib' )
% for author-year bibliography (bmc-mathphys or spbasic)
% a) write to bib file (bmc-mathphys only)
% @settings{label, options="nameyear"}
% b) uncomment next line
%\nocite{label}

% or include bibliography directly:
% \begin{thebibliography}
% \bibitem{b1}
% \end{thebibliography}

\end{backmatter}
\end{document}
