\documentclass[12pt]{article}

\usepackage{sbc-template}

\usepackage{graphicx,url}

\usepackage[brazil]{babel}   

\usepackage[latin1]{inputenc}  

\graphicspath{{./figuras/}}
     
\sloppy

\title{Eventual Consistency and Timeline Consistency\\ Performance in
Geo-replicated Systems}

\author{Mauricio De Diana\inst{1}, Marco Aur�lio Gerosa\inst{1}}

\address{Computer Science Department -- University of S�o Paulo
(USP)\\\email{\{mdediana,gerosa\}@ime.usp.br}}

\begin{document} 

\maketitle

\begin{abstract}

Large-scale web systems use replication among data centers to achieve high
levels of performance and availability. There are known trade-offs between
performance, availability and consistency among replicas. Developers may
analytically grasp such trade-offs from consistency models.  However,
quantitative approaches are necessary for a better understanding of particular
situations.

This work experimentally compared the performance of a storage system using two
different consistency models. Eventual consistency is a more relaxed consistency
model, widely deployed in large-scale systems. By its turn, the less popular
timeline consistency strikes a better balance when requirements over consistency
are more restrictive. The results show that, for some specific workloads and
network conditions, the system exhibits similar performance independently of the
consistency model used. This comparative is helpful in estimating development
costs and during capacity planning of large-scale systems.

\end{abstract}

%% ------------------------------------------------------------------------- %%
\section{Introduction}

Large-scale web systems continuously serve hundreds of thousands of users by
replicating data across thousands of servers. Those servers are spread over
multiple data centers located in different geographical locations, linked by
wide area networks (WANs). However, geo-replication poses a challenge to
maintaining consistency among replicas, mainly due to network latencies and
node / communication failure.

Developers and administrators of such systems seek a balance between
consistency, performance and availability. There are several consistency models
in literature, each of them defining a different set of conditions, guarantees
and trade-offs. A common decision for large-scale systems is to apply more
relaxed consistency models to achieve low latency and high availability.
However, more relaxed consistency models allow inconsistencies to
happen.

Eventual consistency is a more relaxed consistency model which became popular
after the publication of the Dynamo paper \cite{DeCandia2007}. Eventual
consistency guarantees that, updates ceasing, replicas will always converge at
some moment in the future. As long as updates are going on, replicas may be
inconsistent, and clients may read a stale or divergent value depending on which
replica has replied. From that comes the need to implement conflict resolution
mechanisms and compensation actions to handle the results of actions taken based
on inconsistent state. The added complexity negatively affects code
maintainability and system operations.

Some web applications require strict high availability levels. For example,
Amazon needs its users to be able to add items to their shopping cart even if it
leads to some error that must be handled later. They claim that the storage
system backing up the e-commerce offers 99.995\% availability
\cite{DeCandia2007}.

However, eventual consistency is not appropriate for every case. There are
systems that need stronger consistency models, although not linearizability
necessarily.  An example is a bidding application that cannot allow conflicts on
bidding history. If eventual consistency was employed, there could be two
different histories for the same item in case of a network partition allied with
users bidding on that item on each side of the partition. Furthermore, there are
applications that do not necessarily need high levels of availability.

A middle ground between linearizability and eventual consistency is timeline
consistency. It is employed in PNUTS, a database built at Yahoo!
\cite{Cooper2008}. Timeline consistency employs a master replica for each stored
object, this replica is the only one allowed to apply updates. By doing that, it
avoids conflicts among replicas and results in better performance than the one
achieved by expensive mechanisms such as distributed locking. Since updates are
asynchronous, stale values due to latency can happen. However, divergences do
not happen since every replica receive the updates in the same order. Moreover,
the most recent value is known at all times, it is the value stored in the
master replica. Objects are versioned, and read requests carry the required
version, which can be the most recent (served exclusively by the master), any,
or an specific version. The main drawback of timeline consistency is that both
updates and most recent reads become unavailable in case of a
failure preventing the communication between a client and the master.

Timeline consistency may be a better fit than eventual consistency for web
applications where users tolerate some level of unavailability. Nonetheless,
besides availability, performance must also be considered. The main performance
hit in timeline consistency is due to the cost of updates and consistent reads
when the request must traverse a WAN to get to the master in another data
center. However, specific conditions may alleviate this issue, such as high
access locality rates or high read/write ratios. In fact, PNUTS authors observed
up to 85\% locality rate and 94\% of read/write ratio in a study among Yahoo's
applications \cite{Cooper2008, Kadambi2011}. Based on that, they implemented
heuristics that improve timeline consistency performance.

This study compared the performance of a geo-replicated storage system using
eventual consistency and timeline consistency under different workloads and
network conditions. The results make a useful resource for developers and system
administrators estimating system development costs and doing capacity planning.

Section X shows.... [TODO after structure is better defined]

%% ------------------------------------------------------------------------- %%
\section{Eventual Consistency and Timeline Consistency}

Eventual consistency guarantees that replicas will always converge if no
additional updates are made.  While updates happen, replicas might be
inconsistent and clients can access outdated or divergent data -- consequently,
the system must implement algorithms for conflict detection and resolution.  A
solution to lessen the chances of conflicts is the use of quorums, which
drawback is system decreased availability when a specific quorum is not reached
\cite{Vogels2009}. Quorums are commonly defined by N, R and W parameters. N is
the replication factor and represents the number of replicas of a given object.
R / W is the amount of replicas that need to agree on the same value for a
reading / writing be successful. $ N <R + W $ eliminates the possibility of
clients reading inconsistent data, since there is an intersection between the
subsets of replicas for reading and writing.

Some web applications become simpler with a more rigid consistency model. For
example, an auction site cannot tolerate conflicts in the bid history of a
product. In a system using eventual consistency, in case of a failure that
splits the network in two partitions, users in each partition would have their
own vision of the bid history, what is equivalent to two simultaneous auctions
for the same item.

Timeline consistency trades-off availability in some situations over consistency
\cite{Cooper2008}. For each stored object, it allows updates only on one of its
replicas (master replica). Due to asynchronous replication, replicas can have
values outdated due to network latency or failure, however, at any given moment,
the most up-to-date replica is known. In every read, clients choose whether to
accept only the most recent value or not. Furthermore, as the master replica
establishes an update applying ordering to the other replicas, divergences do
not happen and conflict detection and resolution mechanisms are unnecessary. The
main drawback of timeline consistency is that the existence of a master replica
leads to unavailability of writes and up-to-date reads in the event of a failure
that cuts communication with it.

The highest impact factor in the performance of timeline consistency is the fact
that consistent operations that are not made in the master replica's data center
incur the cost of communication over WAN.  However, the authors of PNUTS
indicate that some applications on Yahoo!  exhibit locality of access up to 85
\% and writing / reading ratio of 0.06 \cite{Kadambi2011, Cooper2008}. Given
that, they implemented a heuristic in which the master replica migrates to the
data center that received the last 3 writes. Thus, in an application in which
the amount of readings is much greater than the amount of writes, network
communication costs are low, especially if reads do not necessarily need the
most recent value.

%% ------------------------------------------------------------------------- %%
\section{Experiments Planning}

To understand the performance of each consistency model, an experimental study
was performed. Planning and execution used \cite{Jain1991} as the main
methodological reference.

Three techniques are commonly used in system performance analysis: simulation,
analytical modeling and measurement \cite{Jain1991}. As 33 parameters were
initially considered (see following section), it would be difficult to avoid
loss of accuracy due to simplifications needed to create simulators or models
with that many parameters. Therefore, we chose measurement as technique.

A system had to be chosen as the object of study. As no system implementing both
consistency models has been found, we decided to implement timeline consistency
in Riak, a free software system which implements eventual consistency
\cite{Riak2013}. Besides the new consistency consistency, we also implemented a
partitioning algorithm that ensures that there exist at least one replica of
each object in each data center \footnote{Implementations available at
\url{https://github.com/mdediana/riak_kv} and
\url{https://github.com/mdediana/riak_core}}.

As benchmark, we used Basho Bench, specific to Riak \cite{BashoBench2013}. It
has been adapted to run distributed, with one instance for each data center.

The experiments emulated a WAN with traffic control (tc) / netem. It provides
functionality for emulating network characteristics such as latency and packet
loss. Network settings have been changed in experiments according to
recommendations on Linux optimizations for WAN communication \cite{ESnet2012},
such as using the double of Bandwidth-Delay Product (BDP) as transmission and
reception buffer size.

The experiments were run on Grid'5000, a platform for creating, implementation
and monitoring of parallel and distributed systems experiments
\cite{Grid50002013}.

The study used factorial designs, which consist of a combination of factors in
each experiment comprising the study \cite{Jain1991}. The higher the number of
factors and levels in a study, more resources are needed to its execution.
Moreover, usually a few factors explain the most effects on response. Therefore,
a selection of the most influential ones was performed by the use of
2\textsuperscript{k} experiments. This experiment type is performed with only 2
levels for each factor, a total of 2\textsuperscript{k} experiments, where k is
the number of factors.

%% ------------------------------------------------------------------------- %%
\section{Fixed Parameters}

After defining the type of experimental study, 33 parameters were indicated, 16
of which were fixed due to limited resources or because they are not the focus
study. It is worth noting that LAN parameters refer to those available in the
cluster, while WAN parameters were emulated. The fixed parameters and their
respective values were:

\begin{itemize}

\item \textbf{Cluster:} The experiments used the cluster \textit{sol} in
Grid'5000. The nodes in the cluster have AMD Opteron 2218 2.6 GHz CPU, 4 GB RAM
and 1 Gb/s network cards.

\item \textbf{Storage engine:} By using memory as storage mechanism we avoided
having to consider the effects of disk and disk cache and the interaction
between disk cache and the amount of RAM, so the only I/O effects observed were
due to the network.

\item \textbf{Data center capacity:} Data centers had the same capacity - the
same number of nodes per data center and nodes with the same hardware
configuration.

\item \textbf{Key partitioning algorithm:} Riak's default algorithm was used
(consistent hashing).

\item \textbf{Replication factor ($ N $):} 3 is the value that results in a
reasonable balance between performance, availability and durability in real
applications \cite{DeCandia2007}.

\item \textbf{Migration threshold (for timeline consistency):} 3 is PNUTS
default value \cite{Cooper2008}.

\item \textbf{Interface access:} HTTP, for simplicity of use.

\item \textbf{Log level:} WARN, since exploratory experiments showed loss of
performance when the logging level was INFO.

\item \textbf{Hardware configuration of intermediary network devices:} The
single network device was a FastIron Super X switch, tests have shown there were
no bottlenecks in the switch even in experiments with higher bandwidth
consumption.

\item \textbf{Network topology:} Star, the only available network topology in
the cluster.

\item \textbf{LAN bandwidth:} 1 Gb/s was the bandwidth available through nodes
network cards.

\item \textbf{LAN latency:} $167\mu$s latency cluster measured with 60 samples
of 5-second delayed pings.
% ping -i 5 -c 60 sol-20

\item \textbf{LAN jitter:} $90\mu$s, jitter measured by the same means as
latency. 

\item \textbf{WAN bandwidth:} 100 Mb/s, based on informal study citing this is
the commonly observed  bandwidth between AWS availability zones
\cite{Pujol2012}.

\item \textbf{Number of WAN links:} 1, which results in two data centers used in
the experiments.

\item \textbf{Request arrival rate:} 15 operations/s for each thread of each
benchmark instance.

\end{itemize}

With this, 17 candidates factors still lasted. Three of those were the mode
factor.

%% ------------------------------------------------------------------------- %%
\section{Factor mode}

Three factors have received different treatment during experiments: consistency
model, replication configuration (for eventual consistency) and version required
for reads (for timeline consistency). This was done because the combinations of
these factors define storage system configurations that result in different
local/remote request ratio. Thus, these factors were treated as a single factor
called mode. The used modes were:

\begin{itemize}

\item \textit{ev1}: Eventual consistency with $W$ = 1 and $R$ = 1

\item \textit{ev2}: Eventual consistency with $W$ = 2 and $R$ = 1

\item \textit{any}: Timeline consistency with any version reads

\item \textit{lat}: Timeline consistency with latest version reads

\end{itemize}

Whereas the replication factor was set at 3 and there was at least one replica
on each data center, two situations were possible with respect to the location
of an object replica: one local and two remote or vice versa. Given thats,
\textit{ev1} resulted in local reads and writes. Mode \textit{ev2} resulted in
local reads and half remote, half local writes. Mode \textit{any} resulted in
local reads and the amount of local/remote writes depending on access locality.
Finally, \textit{lat} resulted in both reads  and writes dependent on location.

These modes involve trade-offs beyond performance and consistency. The main is
durability which to \textit{ev2} is higher than for other cases in which write
confirmation of a single replica is sufficient.

There were still 14 candidate factors, too many for the final study.  To reduce
this number, a selection by 2\textsuperscript{k} experiments was performed.

%% ------------------------------------------------------------------------- %%
\section{Factor Selection}

An approach for the selection of factors would group all candidates factors in a
single 2\textsuperscript{k} experimental study. The problem is that even with
only two levels by factor, the final number of experiments would be prohibitive.

The option was to divide them into smaller groups of related factors and run
experiments for each group separately. With that, comparison between factors of
different groups and their interactions is lost, however, since most factors
have shown little influence inside their groups, as shown in the following
subsections, this approach did not present a threat to validity.

Most factors were prone to interactions with network factors. The WAN latency in
particular had been very influential in exploratory studies, later confirmed by
the study of network factors.  Given that, the adopted approach was the use of
WAN latency as a representative of network factors when necessary.

There have been cases where the answers to all experiments of a study were
similar, regardless of the levels used. To handle these cases the coefficients
of variation (CVs) of the responses were calculated to estimate the influence of
that set of factors and interactions as a whole. Thus, a low CV indicated that
none of these factors were influential.

For timeline consistency, only the insertion of objects in the load stage was
not enough for the system to operate at its steady state during the experiments.
This happens because, at the end of the load, each object in the database has
received a single access from each data center, no master replica would have
migrated to this time.  Therefore, a system warmup phase was required after
loading data before each experiment according to the locality of the experiment.

Four intermediate studies were conducted, they are described in the following
subsections.  Whenever mode and locality had to be fixed, they were respectively
at \textit{lat} and 50\%, values that result in a balanced amount of reads and
writes, both local and remote. When necessary, latency was fixed at 100 ms. In
the results, the 10 and 90 percentiles respectively represent local and remote
requests.

%% ------------------------------------------------------------------------- %%
\subsection{System and benchmark size factors}

The number of system nodes and benchmark instances not only influenced the
responses, it also influenced operational issues related to node reservation -
by Grid'5000 rules, the greater the number of reserved nodes, the smaller the
reservation time. Therefore, a study was done to determine the influence of
these factors. The levels used were the following, the value in parentheses
being the identifier of the factor in result tables:

\begin{itemize}

\item Number of system nodes (N): 8:16

\item Number of benchmark instances (B): 2:04

\item Number of threads in each benchmark instance (T): 32 and 64

\end{itemize}

The result of the study is shown in Table
\ref{tab:estudo_para_fatores_de_tamanho_do_sistema}. The system size had greater
influence on the results and the number of benchmark instances and threads were
not negligible, specially considering the interactions between them.
Nevertheless, these factors were disregarded due to the excessive number of
factors.

\begin{table}[ht] \centering \begin{tabular}{|c|c|c|c|c|c|c|c|c|} \hline

Operation & Percentile & N & B & T & NB & NT & BT & NBT\\\hline

read & 10 & 30 & 18 & 22 & 10 & 8 & 7 & 4 \\\hline

read & 90 & 65 & 13 & 15 & 3 & 4 & 0 & 0 \\\hline

write & 10 & 96 & 2 & 1 & 0 & 0 & 0 & 0 \\\hline

write & 90 & 65 & 15 & 13 & 3 & 3 & 0 & 0 \\\hline

\end{tabular} \caption{Study for system size factors.}
\label{tab:estudo_para_fatores_de_tamanho_do_sistema} \end{table}

Given the results, the fixed values were:

\begin{itemize}

\item Number of system nodes: 16

\item Number of benchmark instances: 4

\item Number of threads in each benchmark instance: 32

\end{itemize}

These values were selected because they resulted in the ``lightest''
configuration, avoiding network bottlenecks and system overload. Although
desirable, a greater number of nodes would imply lack of hardware homogeneity
and operational difficulties due to the Grid'5000 rules.

%% ------------------------------------------------------------------------- %%
\subsection{Database factors}

A study was done to select the database size, which affected the node memory
usage and bandwidth comsumption. The levels used were as follows:

\begin{itemize}

\item Number of stored objects (Q): 64,000 and 256,000

\item Size of stored objects (T): 100 and 10,000 bytes

\end{itemize}

The result of the study is shown in Table
\ref{tab:estudo_para_fatores_de_banco_de_dados}. The number of objects did not
affect system performance. Object size did not affect performance of remote
requests, however, they correspond to 100\% of influence regarding local
requests. Nevertheless, the CV of local requests indicated that their influence
was not as great - 19\% for reads and 16\% for writes.

\begin{table}[ht] \centering \begin{tabular}{|c|c|c|c|c|c|c|c|c|} \hline

Operation & Percentile & Q & T & L & QT & QL & TL & QTL\\\hline

read & 10 & 0 & 100 & 0 & 0 & 0 & 0 & 0 \\\hline

read & 90 & 0 & 0 & 100 & 0 & 0 & 0 & 0 \\\hline

write & 10 & 0 & 100 & 0 & 0 & 0 & 0 & 0 \\\hline

write & 90 & 0 & 0 & 100 & 0 & 0 & 0 & 0 \\\hline

\end{tabular} \caption{Study for database factors, L is for latency.}
\label{tab:estudo_para_fatores_de_banco_de_dados} \end{table}

Given the results, the fixed values were:

\begin{itemize}

\item Number of stored objects: 128,000

\item Size of stored objects: 500

\end{itemize}

The warmup time depended on the number of stored objects, thus the lower the
number, the faster the execution of experiments.  Conversely, a too small number
would result in an excessive number of conflicts.  For the size of stored
objects, the value was chosen based on a study of caching systems on Facebook,
which reports that 90\% of the objects are smaller than 500 bytes
\cite{Atikoglu2012}.

%% ------------------------------------------------------------------------- %%
\subsection{Network factors}

Given the purpose of this work, the study for network factors was one of the
most important for the factor selection. The used levels were as follows:

\begin{itemize}

\item WAN latency (L): 100 and 300 ms

\item WAN Jitter (V): 1 and 60\%

\item WAN packet loss rate (P): 0.01 and 0.3\%

\item WAN packet duplication rate (D): 0.05 and 5\%

\item WAN packet reordering rate (O): 0.05 and 5\%

\item TCP variant (T): CUBIC and H-TCP

\end{itemize}

Latency levels were based on a study that reports the latencies between regions
of the Amazon Web Services \cite{Sovran2011}, in which the lowest latency
between data centers in each U.S. coast was 82 ms and the highest was 277 ms
between Ireland and Singapore.
 
Project PingER served as the basis for other factors \cite{PingER2013}. In
January 2013, it shows an average latency of 238.062 ms with standard deviation
142.996 ms, resulting in a jitter of 60\%. The earlier 11 months show similar
figures. The median packet loss rate last year was 0.178\%. The average packet
duplication rate in January 2013 was 0.006\%. The values used in the experiment
for duplication and reordering were higher than those observed by PingER, yet
they did not influence the response.

Both H-TCP and CUBIC were designed with a focus on high bandwidth, high latency
networks (high BDP), and were chosen because they are cited in references on TCP
tuning for WANs \cite{ESnet2012}.

In the emulated network, latency defines the minimum and jitter the maximum
latency can get. For example, when latency is configured as 100ms and varation
as 60\%, the emulator generates latency values between 100 ms and 160 ms. The
generated values followed the normal distribution within the specified range of
latency.

The result of the study is presented in Table
\ref{tab:estudo_para_fatores_de_rede}, columns of interactions between factors
with all cells less than 1\% were suppressed for the sake of space. The
responses of local requests showed CVs of 1\%, so their rows were also
suppressed -- which indicates that WAN does not affect local requests.

\begin{table}[ht] \centering \begin{tabular}{|c|c|c|c|c|c|c|c|c|} \hline

Operation & Percentile & L & V & P & D & O & T & LV\\\hline

read & 90 & 72 & 21 & 1 & 0 & 0 & 0 & 6\\\hline

write & 90 & 69 & 23 & 1 & 0 & 0 & 0 & 6\\\hline

\end{tabular}

\caption{Network factors study.} \label{tab:estudo_para_fatores_de_rede}

\end{table}

Latency, jitter and the first order interaction between them e a intera��o de
make up for close to 100\% of the response. Therefore, the selected levels for
these factors were:

\begin{itemize}

\item WAN latency (ms): 0, 100, 200 and 300

\item WAN jitter (\%): 0 and 60

\end{itemize}

Latency and jitter at zero are equivalent to having the entire system operating
on a LAN. The results obtained in these cases were used as aid in the
interpretation of the results, but were not considered in the final study.

The fixed values of disregarded factors were:

\begin{itemize}

\item WAN packet loss rate: 0\%

\item WAN packet duplication rate: 0\%

\item WAN packet reordering rate: 0\%

\item TCP variant (T): CUBIC

\end{itemize}

%% ------------------------------------------------------------------------- %%
\subsection{Workload factors}

This was one of the most important studies along with the network factors one.
The levels used were as follows:

\begin{itemize}

\item Read / write ratio (R): 2:1 and 10:1

\item Object popularity (P): uniform (the request arrival rate average for each
object is the same) and concentrated (arrival rate follows a Pareto
distribution)

\item Locality (X): 50\% (no locality) and 90\% (90\% of accesses to certain
object coming from a data center and 10\% from the other)

\end{itemize}

As the modes behave differently depending on locality of access, the experiments
were performed for each mode.

The read/write ratio and locality did not use percentiles, they use response
time average of reads and writes instead. That is because the first factor
relates to the composition between reads and writes, while the second alters the
composition between local and remote requests, so these factors do not make
sense in percentiles separated by request type. For example, 50\% locality means
that the 70th percentile is remote requests, while with 90\% locality the same
percentile represents local requests. If the analysis was performed by
percentile, this location information would be lost and locality would never
have any influence.

The result of the study is shown in Table
\ref{tab:estudo_para_fatores_de_carga_de_trabalho}, in which all columns with
all rows less than 5\% were removed for the sake of space. The results for local
requests showed CVs around 2\% for all modes, indicating that none of the
factors influenced local requests.

\begin{table}[ht] \centering \begin{tabular}{|c|c|c|c|c|c|c|c|c|c|} \hline

Mode & R & X & P & L & RX & RL & XL & PL & XPL\\\hline

\textit{ev1} & 19 & 12 & 2 & 31 & 0 & 2 & 6 & 6 & 8\\\hline

\textit{ev2} & 50 & 0 & 0 & 39 & 0 & 11 & 0 & 0 & 0\\\hline

\textit{any} & 25 & 30 & 0 & 19 & 9 & 6 & 8 & 0 & 0\\\hline

\textit{rec} & 0 & 53 & 0 & 34 & 0 & 0 & 13 & 0 & 0\\\hline

\end{tabular} \caption{Study for workload factors, L is latency.}
\label{tab:estudo_para_fatores_de_carga_de_trabalho}

\end{table}

As expected, locality and latency influenced responses in general. The impact of
popularity of objects is virtually nil. Although some modes were apparently
impacted by read/write ratio, this impact was a consequence of the relationship
between local and remote requests. For \textit{ev1}, both reads and writes are
local and the read/write ratio and their interactions with other factors impact
little this mode.  For \textit{rec}, reads and writes are local or remote
depending on locality and read/write ratio does not impact this mode. For
\textit{ev2}, all reads are local and half of writes are remote, therefore when
the read/write ratio changes, the ratio between requests local and remote
changes proportionally -- as expected, this mode is impacted read/write ratio.
The same observation holds for \textit{any}, which has all reads local and
writes depending on locality, and also impacted the read/write ratio. The
read/write ratio would likely influence the response if the storage mechanism
was disk instead of memory, since writes would be affected by the disk write
time disk, while reads could be faster because part of them could be served from
the disk cache.

Therefore, only locality was selected as factor:

\begin{itemize}

\item Locality (\%): 50 and 90

\end{itemize}

The fixed values of disregarded factors were:

\begin{itemize}

\item Read/write ratio: 2:1

\item Object popularity: uniform

\end{itemize}

%% ------------------------------------------------------------------------- %%
\section{Final Study}

The final study consisted of a total of 64 experiments with the factors selected
by the 2\textsuperscript{k} studies, shown in
Table~\ref{tab:fatores_e_niveis_do_estudo_final}. The experiments used a number
of samples such that the confidence level was 99\% and accuracy was 1\%
\cite{Jain1991}. The only exception where the number of samples was smaller than
the expected was \textit{rec} -- in this case, the same confidence level was
adopted and the accuracy was 2\%. Two replications of the study were made to
estimate the variability of the experiments. The average CVs of the experiments
was 1\% for reads and 0.8\% for writes.

\begin{table}[ht] \centering \begin{tabular}{|l|c|c|} \hline

\multicolumn{1}{|c|}{Factor} & \multicolumn{1}{|c|}{Levels} &
\multicolumn{1}{|c|}{Number of levels}\\\hline

Mode & \textit{ev1}, \textit{ev2}, \textit{any} e \textit{rec} & 4\\ \hline

WAN latency (ms) & 0, 100, 200 and 300 & 4\\\hline

WAN jitter (\%) & 0 and 60 & 2\\\hline

Locality & 50\% and 90\% & 2\\\hline

\end{tabular}

\caption{Factors and levels in the final study.}
\label{tab:fatores_e_niveis_do_estudo_final}

\end{table}

The results for latencies 100~ms, 200~ms and 300~ms have the same behavior, so
the option was to do the analysis to one of them only. The boxplot to network
latency of 200~ms is shown in Figure \ref{fig:boxplot_dos_tempos_de_resposta}.
The cases in which the box does not appear indicate that all requests delimited
by the boxplot whiskers were local. The same analysis was done with maximum load
instead of 15 operations/s per thread. As
Figure~\ref{fig:boxplot_dos_tempos_de_resposta_para_carga_maxima} shows, all
modes showed an increase in response times, but kept the same behavior.

\begin{figure}[ht] \centering

\includegraphics[width=0.7\textwidth]{boxplot200.png}

\caption{Boxplot of response times for 15 operations/s per thread.}
\label{fig:boxplot_dos_tempos_de_resposta} \end{figure}

\begin{figure}[ht] \centering

\includegraphics[width=0.7\textwidth]{boxplot200_max.png}

\caption{Boxplot of response times for maximum load per thread.}
\label{fig:boxplot_dos_tempos_de_resposta_para_carga_maxima} \end{figure}

In the case of reads and 50\% locality, only \textit{lat} shows remote requests.
A small portion of those reads have response time smaller than the network
latency, which is explained by the jitter of up to 60\%.

For reads and locality of 90\%, \textit{lat} is benefitted, but still shows some
remote requests (outliers in the graph). In the event of writes and locality of
50\%, \textit{lat} performs slightly better than \textit{ev2} and \textit{any}.
It happens because the system handles a smaller load in this mode than the
others, since reads for \textit{lat} are slower. This fact is evidenced by the
throughput of each mode \textit{lat}, \textit{ev2} and \textit{any},
respectively, 594, 1072, 941 operations/s.

For writes and 90\% locality, \textit{lat} and \textit{any} have better
performance due to the locality, but still show remote requests (outliers in the
graph). By its turn, \textit{ev2} still showsthe same pattern as for 50\%
locality.

The modes showed the expected behavior. The hypothesis that timeline consistency
is competitive in performance with eventual consistency is confirmed for the
case where locality is high and especially when readings of ``any version'' are
employed.

%% ------------------------------------------------------------------------- %%
\section{Threats to Validity}

Some parameters were fixed and others found influential were disregarded.  Thus,
studies that use other values for the parameters or consider other factors may
end at different results. This is particularly true for the number of system
nodes, which showed as highly influential.  In addition, different level ranges
may lead to other results \cite{Jain1991}.

The experiments do not take node failures into account.  Experiments with the
system operating in a failure mode (from the failure of a singe node to a whole
datacenter) must get to results different than the shown here. Experiments like
that were not performed due to the research resource limitations.

%% ------------------------------------------------------------------------- %%
\section{Related Work}

A common approach in the distributed systems literature is the proposal of a new
concept and the implementation of a system that uses this concept followed by an
analysis of its performance. Both papers on Dynamo \cite{DeCandia2007} and on
PNUTS \cite{Cooper2008} show performance analyzes, and the second only is a
workload-based analysis. Another paper on PNUTS analyzes bandwidth consumption
under different replication policies for comunication over WAN
\cite{Kadambi2011}.

Several papers present performance analysis of storage systems using replication
over WANs. In most cases, these systems goals are to show ideas other than the
efficiency of their choice of consistency model. COPS uses causal+ consistency
that is similar to causal consistency with some more guarantees, and implements
transactions \cite{Lloyd2011}. Scatter proposes an scalable architecture at the
same time as strong consistency \cite{Glendenning2011}. Windows Azure provides a
cloud storage system with strong consistency \cite{Calder2011}. Megastore uses
Paxos to implement strong consistency \cite{Baker2011}. None of these works
presents comparisons with other systems or other consistency models. As they do
not use a common benchmark nor environment, it is difficult to compare them
them.

Cassandra's different consistency options and their respective availability and
performance were analyzed in \cite{Beyer2011}, concluding that settings with
more rigid consistency models performs worse. Performance and availability of
master-slave replication and chain replication are compared in
\cite{vanRenesse2004}, each of them using strong consistency and eventual
consistency. None of these studies considers operation over WAN or different
workloads.

A study proposal closer to this work is the comparison between Cassandra, HBase,
PNUTS and sharded MySQL under different workloads \cite{Cooper2010}. The results
are a comparison between these systems, but do not say much about their
consistency models, since each have different architectures and configuration
options. Moreover, tests were made over a LAN only.

%% ------------------------------------------------------------------------- %%
\section{Conclusions}

This study compared the performance of a single storage system, operating over a
WAN, using two different consistency models. Furthermore, it showed results
about the influence of different factors and their interactions on the system
performance. Timeline consistency showed competitive with eventual consistency
when write locality is high and when reads accept stale data.

The main advantages of timeline consistency over eventual consistency are the
guarantee that conflicting updates cannot happen and that the location of the
most recent value is known -- the master. An interesting scenario for its use is
the case where the application tolerates staleness in most reads, but in a few
it needs the most recent value. Timeline consistency main drawbacks are that the
unavailability of master replica prevents writes and ``most recent'' reads from
happening and that even being competitive in terms of performance, it has
relatively high response teime variability. For applications where response time
requirements are on the 99.9 percentile, such as Amazon \cite{DeCandia2007},
timeline consistency is not appropriate.

%% ------------------------------------------------------------------------- %%
\section{Agradecimentos} 

Experiments presented in this paper were carried out using the Grid'5000
experimental testbed, being developed under the INRIA ALADDIN development action
with support from CNRS, RENATER and several Universities as well as other
funding bodies.

\section{Refer�ncias} \bibliographystyle{sbc} \bibliography{bibliografia}

\end{document}
